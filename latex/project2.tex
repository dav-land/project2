\documentclass[12pt]{article}   % use documentclass amsart too if you want
\usepackage{amsmath,amsthm,amssymb}
\usepackage[margin=1in]{geometry}
\usepackage{color}
\usepackage{hyperref}
\usepackage{graphicx}
\usepackage{fancyhdr} % COMMENT THIS OUT TO TURN OFF FANCY HEADERS
\usepackage{verbatim}
\usepackage{setspace}

\hypersetup{
  colorlinks= true, %Colours links instead of ugly boxes
  urlcolor   = blue, %Colour for external hyperlinks
  linkcolor  = blue, %Colour of internal links
  citecolor  = blue %Colour of citations
}





\newtheorem{theorem}[equation]{Theorem}
\newtheorem{lemma}[equation]{Lemma}
\newtheorem{corollary}[equation]{Corollary}
\theoremstyle{definition}
\newtheorem{exercise}[equation]{Exercise}

\newtheorem{example}[equation]{Example}
\newtheorem{definition}[equation]{Definition}
\newtheorem{question}[equation]{Question}
\newtheorem{remark}[equation]{Remark}

\numberwithin{equation}{section}

%@@@@@@@@@@@@@@@@@@@@@@@@@@@@@@@@@@@@@@@@@@@@@@@@@@@@@@@@@@@@@@@@@@@@@@@@@@@@@@@@@@@@@@@@@

\begin{document}
\parskip10pt
\parindent0pt
\baselineskip15pt
\doublespacing

\title{APPM 2350 Project 2: Hiking}
\author{Davis Landry (section 223), Mahalie Hill (section 223)\\Mia Miller (section 213) \\ Jonathan Kish, Maribeth Oscamou}

%I think we should add a photo here for the cover page
\pagestyle{fancy}
\renewcommand{\sectionmark}[1]{\markright{#1}{}}

\fancyhf{}

\rhead{\fancyplain{}{\thepage}} % predefined ()
\lhead{\fancyplain{}{\rightmark }} % 1. sectionname, 1.1 subsection name etc
%\cfoot{\fancyplain{}{\thepage}}

\maketitle
\newpage
%\setcounter{page}{2}
\tableofcontents
\newpage
%@@@@@@@@@@@@@@@@@@@@@@@@@@@@@@@@@@@@@@@@@@@@@@@@@@@@@@@@@@@@@@@@@@@@@@@@@@@@@@@@@@@@@@@@@
\newpage
\lhead{Landry, Hill, Miller Project 1}
%\setcounter{page}{2}

\section{Introduction} \label{APPM2350proj01sec01}

\quad Welcome to Colorado! Before we embark on this epic journey through the Lagrange Mountain range, we have compiled a travel packet for you to understand the trail system we will be hiking. We promise, after reading this trail analysis, you'll feel equiped to conquer this trail.

%@@@@@@@@@@@@@@@@@@@@@@@@@@@@@@@@@@@@@@@@@@@@@@@@@@@@@@@@@@@@@@@@@@@@@@@@@@@@@@@@@@@@@@@@
%\setcounter{page}{4}
\section{Basic Trail Information} \label{APPM2350proj01sec02}

\quad The Lagrange loop is located a short drive from Boulder, CO. It is a \textcolor{red}{DISTANCE OF TRAIL} hike, with \textcolor{red}{FEET OF ELEVATION CHANGE}. You better be in great shape, because we expect to finish this hike in roughly 1 hour 34 minutes. The hike consists of jaw-dropping views of Lake Mochi, with views of the sheer sides of Mount Adamore in the background. This trail is primarily used for speed hiking and trail running because the Honey Badger at Curtis Pass enjoys giving chase to trail users, a great endurance training regime for the super athletes of Colorado. Dogs are not allowed due to the Honey Badger, so leave your furry friends at home.

%@@@@@@@@@@@@@@@@@@@@@@@@@@@@@@@@@@@@@@@@@@@@@@@@@@@@@@@@@@@@@@@@@@@@@@@@@@@@@@@@@@@@@@@@@@@@@@@@@@@@@@@@@@@@@@@@@@@@@@
%\setcounter{page}{7}
\section{The Specs}
\label{APPM2350proj01sec03}

\quad Pictured in \textcolor{red}{3D FIGURE WITH TRAIL}, you can see the general trail outline. The trail is a two-way path, so it can be taken from either direction from the trailhead. It is highly recommended that those with bad knees go counterclockwise around the loop, due to the steep section of trail on Mount Adamore.

\quad Looking at \textcolor{red}{TOPO FIGURE}, the general steepness of the trail can be assumed. Note that at Curtis pass, the trail travels uphill \textcolor{red}{I THINK?}. This is where our friend the Honey Badger calls home, and he despises the frequent visitors on his doorstep. Beware, because we will have to pick up our pace through here to avoid him. The trail slope is fairly steep, at \textcolor{red}{HONEY BADGER SLOPE RESULT HERE}, and the elevation change is \textcolor{red}{HONEY BADGER ELELVATION CHANGE}. Hopefully those sea level lungs of you out-of-staters can handle the stress.
\textcolor{red}{INSERT PHOTO OF ANGRY HONEY BADGER}

\quad The steepest section of trail along the Lagrange Loop is \textcolor{red}{TIME VALUE FOR GREATEST RATE OF CHANGE} \textcolor{red}{HOURS OR MINUTES} into the hike, so make sure to save water and snacks for this section. The most leisurely section of the hike, with the smallest elevation change \textcolor{red}{TIME VALUE FOR SMALLEST RATE OF CHANGE} \textcolor{red}{HOURS OR MINUTES} into the hike.
\quad Although that is our steepest climb, and flattest section, the trail is highly variable, so we will be embarking on a number of climbs throughout. If you are wondering where those other climbs are in relation to the time we will spend out hiking, take a look at \textcolor{red}{FIGURE WITH PLOT OF RATE OF CHANGE OF ELEVATION OVER WHOLE PATH}, so you know how many snacks and how much water to save. Hydrate or die folks!

\quad Mount Adamore is the highest peak in the Lagrange Range, with an elevation of 8,100.8ft. The path does not go to the peak, so the maximum elevation we will reach 7,934ft, easily accomplishable with lungs acclimated to lower elevations. No altitude sickness today kids! We will be staying well under 9,000ft.

\quad \textcolor{red}{INSERT PRETTY MOUNTAIN LAKE PHOTO ABOVE THIS PARAGRAPH} Nothing is more of a mountain adventure than swimming in an alpine lake! Lake Mochi sits in the valley created by the Lagrange Range, a pristine oasis surrounded by beautiful mountains. Note that the trail does not go directly to the lake, but a little bit of bush-wacking from the trailhead can get us there in a few minutes without too much elevation change. View \textcolor{red}{FIGURE OF 3D PLOT WITH MOCHI} to get a sense of how far it is to the lake based on the length of the rest of the hike.
%@@@@@@@@@@@@@@@@@@@@@@@@@@@@@@@@@@@@@@@@@@@@@@@@@@@@@@@@@@@@@@@@@@@@@@@@@@@@@@@@@@@@@@@@
%\setcounter{page}{12}
\section{Trail Lore} \label{APPM2350proj02sec05}

\quad These beautiful peaks encase a whole lot of rock. Actually, about \textcolor{red}{ROCK VOLUME} $ft^3$, to be more precise if we consider the trail and the valley the trail surrounds. For the rock above 7,000ft, there is a legend that Clairautnium can be found in scarce quantities. Roughly \textcolor{red}{MASS OF CLAIRAUTNIUM} grams of Clairautnium is estimated to be encased in the mountain range. Considering there is \textcolor{red}{VOLUME OF ROCK above 7000???} rock, and a density of Clairautnium of \textcolor{red}{CLAIRAUTNIUM DENSITY}, we are only \textcolor{red}{LIKELIHOOD OF FINDING CLAIRAUTNIUM how tf do we calculate this...?} likely to come across any on the hike... So cross your fingers and hope we find some!

\quad Why would people care about Clairautnium you ask? Well legend has it that \textcolor{red}{SOME FUNNY LEGEND I'M NOT CREATIVE SORRY}
%@@@@@@@@@@@@@@@@@@@@@@@@@@@@@@@@@@@@@@@@@@@@@@@@@@@@@@@@@@@@@@@@@@@@@@@@@@@@@@@@@@@@@@@@
%\setcounter{page}{13}
\section{Summary} \label{APPM2350proj02sec06}

\quad Hopefully you are intrigued enough to come adventure throughout the Lagrange Range. Remember to pack enough snacks and water to maintain high morale and energy! Hungry friends are not happy friends... So make sure to bring a little extra for the Honey Badger. At the speed recommended to conquer this trail, hopefully you have been training for the steep climbs. Let's get into the Colorado mindset and take on the trails!

%@@@@@@@@@@@@@@@@@@@@@@@@@@@@@@@@@@@@@@@@@@@@@@@@@@@@@@@@@@@@@@@@@@@@@@@@@@@@@@@@@@@@@@@@
%\setcounter{page}{14}
\section{Lagrange Loop Not Challenging Enough?} \label{APPM2350proj02sec07}

\textcolor{red}{Create Extra Credit Mountain Range Here :)}

%@@@@@@@@@@@@@@@@@@@@@@@@@@@@@@@@@@@@@@@@@@@@@@@@@@@@@@@@@@@@@@@@@@@@@@@@@@@@@@@@@@@@@@@@
%\setcounter{page}{15}
\section{Appendix} \label{APPM2350proj02sec08}

\textcolor{red}{Include calculations here, so put Mathematica, etc. in this section}

%@@@@@@@@@@@@@@@@@@@@@@@@@@@@@@@@@@@@@@@@@@@@@@@@@@@@@@@@@@@@@@@@@@@@@@@@@@@@@@@@@@@@@@@@


\end{document}
