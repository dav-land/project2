\documentclass[12pt]{article}   % use documentclass amsart too if you want
\usepackage{amsmath,amsthm,amssymb}
\usepackage[margin=1in]{geometry}
\usepackage{color}
\usepackage{hyperref}
\usepackage{graphicx}
\usepackage{fancyhdr} % COMMENT THIS OUT TO TURN OFF FANCY HEADERS
\usepackage{verbatim}
\usepackage{setspace}

\hypersetup{
  colorlinks= true, %Colours links instead of ugly boxes
  urlcolor   = blue, %Colour for external hyperlinks
  linkcolor  = blue, %Colour of internal links
  citecolor  = blue %Colour of citations
}





\newtheorem{theorem}[equation]{Theorem}
\newtheorem{lemma}[equation]{Lemma}
\newtheorem{corollary}[equation]{Corollary}
\theoremstyle{definition}
\newtheorem{exercise}[equation]{Exercise}

\newtheorem{example}[equation]{Example}
\newtheorem{definition}[equation]{Definition}
\newtheorem{question}[equation]{Question}
\newtheorem{remark}[equation]{Remark}

\numberwithin{equation}{section}

%@@@@@@@@@@@@@@@@@@@@@@@@@@@@@@@@@@@@@@@@@@@@@@@@@@@@@@@@@@@@@@@@@@@@@@@@@@@@@@@@@@@@@@@@@

\begin{document}
\parskip10pt
\parindent0pt
\baselineskip15pt
\doublespacing

\title{APPM 2350 Project 2: Hiking}
\author{Davis Landry (section 223), Mahalie Hill (section 223)\\Mia Miller (section 213) \\ Jonathan Kish, Maribeth Oscamou}

%I think we should add a photo here for the cover page
\pagestyle{fancy}
\renewcommand{\sectionmark}[1]{\markright{#1}{}}

\fancyhf{}

\rhead{\fancyplain{}{\thepage}} % predefined ()
\lhead{\fancyplain{}{\rightmark }} % 1. sectionname, 1.1 subsection name etc
%\cfoot{\fancyplain{}{\thepage}}

\maketitle
\newpage
%\setcounter{page}{2}
\tableofcontents
\newpage
%@@@@@@@@@@@@@@@@@@@@@@@@@@@@@@@@@@@@@@@@@@@@@@@@@@@@@@@@@@@@@@@@@@@@@@@@@@@@@@@@@@@@@@@@@
\newpage
\lhead{]Landry, Hill, Miller Project 1}
%\setcounter{page}{2}

\section{Introduction} \label{APPM2350proj01sec01}

\quad Welcome to Colorado! Before we embark on this epic journey through the wonders of the Lagrange Mountain range, we have compiled a travel packet for you to understand the trail system we will be hiking. We promise, after reading this trail analysis, you'll feel equiped to conquer this trail.

%@@@@@@@@@@@@@@@@@@@@@@@@@@@@@@@@@@@@@@@@@@@@@@@@@@@@@@@@@@@@@@@@@@@@@@@@@@@@@@@@@@@@@@@@
%\setcounter{page}{4}
\section{Basic Trail Information} \label{APPM2350proj01sec02}

\quad The Lagrange loop is located a short drive from Boulder, CO. It is a \textcolor{red}{DISTANCE OF TRAIL} hike, with \textcolor{red}{FEET OF ELEVATION CHANGE}. You better be in great shape, because we expect to finish this hike in roughly 1 hour 34 minutes. The hike consists of jaw-dropping views of Lake Mochi, with views of the sheer sides of Mount Adamore in the background. This trail is primarily used for speed hiking and trail running, due to the Honey Badger that enjoys giving chase to trail users, a great endurance training regime for the endurance athletes of Colorado. Dogs are not allowed due to the Honey Badger at Curtis pass, so leave your furry friends at home.

%@@@@@@@@@@@@@@@@@@@@@@@@@@@@@@@@@@@@@@@@@@@@@@@@@@@@@@@@@@@@@@@@@@@@@@@@@@@@@@@@@@@@@@@@@@@@@@@@@@@@@@@@@@@@@@@@@@@@@@
%\setcounter{page}{7}
\section{Applications}
\label{APPM2350proj01sec03}


%@@@@@@@@@@@@@@@@@@@@@@@@@@@@@@@@@@@@@@@@@@@@@@@@@@@@@@@@@@@@@@@@@@@@@@@@@@@@@@@@@@@@@@@@
%\setcounter{page}{12}
\section{Calculations} \label{APPM2350proj01sec05}


%@@@@@@@@@@@@@@@@@@@@@@@@@@@@@@@@@@@@@@@@@@@@@@@@@@@@@@@@@@@@@@@@@@@@@@@@@@@@@@@@@@@@@@@@
%\setcounter{page}{13}
\section{Results} \label{APPM2350proj01sec06}


%@@@@@@@@@@@@@@@@@@@@@@@@@@@@@@@@@@@@@@@@@@@@@@@@@@@@@@@@@@@@@@@@@@@@@@@@@@@@@@@@@@@@@@@@
%\setcounter{page}{14}
\section{Conclusion} \label{APPM2350proj01sec07}


%@@@@@@@@@@@@@@@@@@@@@@@@@@@@@@@@@@@@@@@@@@@@@@@@@@@@@@@@@@@@@@@@@@@@@@@@@@@@@@@@@@@@@@@@
%\setcounter{page}{16}
\section{Appendix} \label{APPM2350proj01sec08}



\end{document}
